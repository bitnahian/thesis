\phantomsection\addcontentsline{toc}{chapter}{Abstract} % Don't remove me!
\chapter*{\centering{\Large Abstract}}
% A brief introduction to the topic that you're investigating.
% Explanation of why the topic is important in your field/s.
% Statement about what the gap is in the research.
% Your research question/s / aim/s.
% An indication of your research methods and approach.
% Your key message.
% A summary of your key findings.
% An explanation of why your findings and key message contribute to the field/s.
Commercial cloud providers for ready-to-use scale-out database solutions are continually confronted by the growing need to overcome the performance bottlenecks of Big Data Analytics. Despite continuous improvements in distributed query processing, customers \textit{still} face the important task of deciding the physical placement of their data across distributed machines. Big data analytics typically involves complex join queries over two or more tables. Such join processing is expensive in a distributed setting both because large amounts of data must be read from disk or memory, and because of data shuffling across the network. Database administrators tend to rely on simple heuristics to determine the physical placement of their data, which can often lead to sub-optimal partitioning schemes, and in turn create performance bottlenecks.

In this Honours thesis, we explore the idea of automating the data placement decisions using a machine learning-based (ML-based) partitioning advisor using Deep Reinforcement Learning with different potential query workloads to 'learn' the best physical data placement in terms of data partitioning. Most importantly, (i) we introduce a query featurisation step to normalise query workloads; (ii) incorporate the \textbf{state} of the database's query optimiser in the online training phase for the advisor to observe; and (iii) enforce the state model and action search space to include composite key partitioning strategies.

We evaluate our learned partitioning advisor on a synthetic query benchmark against an existing Deep Reinforcement Learning based technique and heuristics. In our evaluation, we show that not only do our implemented optimisations allow the learner to find partitioning schemes where the query benchmarks run faster than with existing approaches, we also show that our approaches converge faster than existing Deep Reinforcement Learning techniques. This is especially important in terms of improving the practical applications of Deep Reinforcement Learning based techniques for data partitioning tasks after we make a case for integrating the query optimiser as part of the learning process in order to find better and faster solutions.

% Once a corresponding data placement model has been learned, the advisor can then be used as an automatic component of a distributed database to decide automatically given the current database schema and the workload how to place the data for best performance. In our approach, we contribute 